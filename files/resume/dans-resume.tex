






\iffalse


Welcome to the .tex side fellow developer!!!

This is the uncompiled-latex version of my resume which is different then my pdf resume in many ways. I have coded this tex file to be as modular as it can be, meaning that instead of sending tailored resumes to different companies, I have made everything that I want to show off available on this file, and whenever I want to apply to a certain job, I leave the projects that I want, comment out the rest and compile!

So you should be able to see a summary of almost all of my projects on this file. I am also very candid about many different points and have added comments here and there to further clarify many things.

if you want to compile on your own:
please use "xelatex", sometimes pdflatex or other compilers' outputs end up not looking the way the resume was intended to be.


\fi







\documentclass[]{dans-resume}
\begin{document}


% Title
\namesection{Daniel}{Razavi Azar Khiavi}{
\href{mailto:daniel.razaviazarkhiavi@mail.utoronto.ca}
{daniel.razaviazarkhiavi@mail.utoronto.ca} | 647.960.7621 | Aurora, Canada
}

% Column 1
\begin{minipage}[t]{0.30\textwidth}

% Objective
\section{Statement}
Self-driven Student with an intrest towards many branches of software development and an abtitude for new technologies. Work exceptionally well in group projects, and always deliver the best quality of work. Highly adaptable to new environments and disciplines.
\sectionsep

% Education
\section{Education}
\subsection{University of Toronto}
\descript{B.S in Computer Science, Statistics and Mathematics}
\location{Expected May 2020 | Mississauga, Ontario}
Sessional GPA: 3.5
\sectionsep

% Links
\section{Links}
Website: \href{https://danielrazavi.github.io/}{\custombold{danielrazavi.github.io}} \\
LinkedIn: \href{https://www.linkedin.com/in/danielrazavi}{\custombold{Daniel Razavi}} \\
% Github:// \href{https://github.com/DanielRazavi}{\custombold{Daniel Razavi}}\\

% Github is included on my website but here it is, just incase ^^
\sectionsep

% Computer Science courses that have been taken so far are (from most recent/current to oldest):
% Programming on the Web                        (CSC 309) (current)
% Neural Nets and Deep Learning                 (CSC 421) (Current)
% Intro to Databases                            (CSC 343)
% Machine Learning and Data                     (CSC 411)

% Data Structures and Analysis                  (CSC 263)
% Introduction to Software Engineering          (CSC 301)
% Introduction to Artificial Intelligence       (CSC 384)
% Operating Systems                             (CSC 369)
% Principles of Programming Languages           (CSC 324)

% Computer Organization                         (CSC 258)
% Software Tools and Systems Programming        (CSC 209)
% Software Design                               (CSC 207)
% Introduction to the Theory of Computation     (CSC 236)

% Communication Skills for Computer Scientists  (CSC 290)
% Introduction to Computer Science              (CSC 148)
% Introduction to Computer Programming          (CSC 108)

% Some of the Courses I have taken
\section{Coursework}
Data Structures and Analysis, \\
Operating Systems, \\
Intro. to Databases,\\
Intro. to Software Engineering, \\
Intro. to Artificial Intelligence \\
Programming on the Web Development (In Progress), \\
Neural Networks and Deep Learning (In Progress)\\
\sectionsep

% Skills
\section{Skills}

% Programming Languages
\subsection{Languages}
Java \textbullet{} Python \textbullet{} C \textbullet \LaTeX\ \textbullet R \\
HTML \textbullet{} CSS \textbullet{} SQL \textbullet{} XML\\
Racket \textbullet{} Haskell \textbullet{} Prolog\\
\sectionsep

% Software
\subsection{Software}
Spyder \textbullet{} Eclipse \textbullet{} Pycharm \\
Visual Studio \textbullet{} Atom \textbullet{} Vim \\
Nano \textbullet{} Github \textbullet{} Bitbucket \\
Android Studio
\sectionsep

% Tools
\subsection{Tools and}
\subsection{Operating Systems}
Shell  \textbullet{} GIT \textbullet{} GDB  \textbullet{}
Scrum \textbullet{} Linux \\
Windows \textbullet{} MacOS \\
\sectionsep
\end{minipage}
\hfill
% Column 2
\begin{minipage}[t]{0.66\textwidth}

% Experience
\section{Experience}

\runsubsection{University of Toronto}
\descript{| Teacher's Assistant}
\location{September 2018 – December 2018 | Mississauga, ON}
\vspace{\topsep} % Hacky fix for awkward extra vertical space
\begin{tightemize}
\item Faciliated group exercises and challenged students to apply persented material from class,
in their personal perojects.
\item Instructed two classes of CSC236 (Intro. to the Theory of
Computation) to second years students, under the direction and guidance of course instructors.
\item Learned the basics of web development and created a website that can help students
recieve class slides and extra resources.
\end{tightemize}


% Worked with the amazing professor Illir Dema, who was the course coordinator of CSC236. He really took a chance by making me as a TA, and I am glad that he was happy with his decision throughout the time that I worked with him.


\runsubsection{UTM Hack Lab}
\descript{| GIT Workshop Instructor}
\location{January 2018/2019 – | Mississauga, ON}
\begin{tightemize}
  \item Assissted Computer Science students to fully learn GIT.
  \item Supported students to incorporate GIT in their assignments
  and/or personal projects.
  \item Eliminated common missunderstandings in regards to GIT by making sure every student
  creates and configures a repo of their own, on their own device.
\end{tightemize}

% UTM Hack Lab is an unofficial computer science club by students, for students, where those who are great at a particular area of software development, have the ability to host workshops for others. I had the pleasure of hosting two workshops regarding GIT, on January of 2018 and 2019. Went over the basics of GIT with mostly first year and second year students where I made sure that they are aware of its functionality and potential in their own personal projects. The idea came from that fact that some first year students were using google drive to share code with eachother.

\runsubsection{ALDO GROUP}
\descript{| Sales Associate }
\location{July 2016 – August 2017 | Newmarket, ON}
% \vspace{\topsep} % Hacky fix for awkward extra vertical space
\begin{tightemize}\item Cashier and Customer Care at Aldo, Upper Canada Mall.
  Providing excellent customer service to vistors by answering questions, training new members
  and handling exchanges/resturns.
  \item Learning and adapting in short periods of time due to constant change in sales.
\end{tightemize}

% The highest level of retail that I have worked at. I have also worked for McDonald's and the Children's Place.

% Projects
\section{Projects}

\subsection{Personal Website}
\vspace{\topsep}
\begin{tightemize}\item Made a personal website to learn new web development technologies.
\item Originally designed as a resource for my students but currently is used to  showcase
my projects to visiters and potential emplyers.
\item Created using only HTML and CSS (Currently enrolled in Web Development - More
technologies on the way)
\end{tightemize}
\sectionsep

%  The inspiration came to me on the first day of my TA job where I was really nervous and I was trying to sell to my students the fact that I am a great teacher. After teaching the class smoothly with the powerpoint presentation that I had worked hours to prep and create, I presented the students with a QR scanner so that they don't have to worry about taking notes, and for them to just listen to the material. I decided to make the website due to the fact that one of the students didn't have a smartphone and had to ask me if I can give him the direct link; figured a website is a better solution. Over time it has become more then that.

\runsubsection{Operating System}
\descript{| CSC369 Assignments }
% \vspace{\topsep}
\begin{tightemize}
  \item In a team of two, worked on low level c language in order to manipulate
  kernel modules and to hijack syscall functions to my own custom functions.
  \item Implemented four different page replacement algorithms: FIFO , Clock
   exact LRU, and OPT
  \item Explored the implementation of ext2 file system and written tools to
  modify ext2-format virtual disks. Tools such as ls, cp, mkdir, rm and restore.
  % \item Applied tools such as GIT to synchronize the team and Gdb to fully debug problems in
  our multi-threaded projects.
\end{tightemize}
\sectionsep

% Not one project but several projects that I worked on in my Operating Systems class (CSC3 69). Became familiar with many aspects of an OS and also had to learn to become comfortable with GDB for debugging.

\iffalse
\runsubsection{Analysis of Food Value On Campus}
\descript{| STA304 Assignment}
% \vspace{\topsep}
\begin{tightemize}\item Took a systematic sample of each food team (previously
  defined) which yielded us (a team of 7) our sampling frame.
  \item Found the ratio of calories per dollar and used that in a program
  (written in R) to determine which food item had the highest ratio.
  \item Did this for all food categories that we had defined
  and all locations on campus.
\end{tightemize}
\sectionsep
\fi

% I am also taking a minor in Statistics and this project was a way for me to utilize proper sampling methods that can give unbiased results. Very basic usage of R from it's histogram graphing tools to importing and utilizing data from csv files.

\iffalse
\runsubsection{Pac-man AI Agents}
\descript{| CSC384 Assignments}
% \vspace{\topsep}
\begin{tightemize}\item Designed custom BFS, DFS and A-Star Search Patterns and
   their respective Heuristics in order to make pac-man smarter at making
   decisions.
  \item Integrated minimax with alpha beta pruning to make pac-man at high alert at
  all times and making sure it always makes the best possible decision.
  \item Created a expectimax agent for pac-man to not assume the worst case
  scenerio, rather the expected move the ghosts will make.
\end{tightemize}
\sectionsep
\fi


\iffalse
\runsubsection{Paint Project}
\descript{| CSC207 Assignment}
\vspace{\topsep}
\begin{tightemize}
    \item Made a Java program which is a simplified version of the windows' paint
    program by using scrum framework amongst a team of 4.
    \item Offered users multiple drawing tools from basic lines to geometrical
    shapes.
    \item Used factory design pattern for shapes and used MVC design pattern to make the
     code easy to follow. By using these design patterns, the project became
     more modular and more Object Oreiented.
    \item Used Observer design pattern to make the process seamless on the view
    side of the MVC design pattern.
\end{tightemize}
\sectionsep
\fi

\runsubsection{R.Edu Project}
\vspace{\topsep}
\begin{tightemize}
    \item An Android application developed to create a virtual place for students
    to rate certain professors and/or courses.
    \item The role of UI and UX developer in an Agile environment (specifically scrum)
     in a team of 4.
    \item Used MVC design pattern to make the code easy to follow and to assign tasks.
    \item Utilized Observer design pattern to make the process seamless on the view
    side of the MVC design pattern.
\end{tightemize}
\sectionsep

% A course project where to be fair, shouldn't count as one since the entire class was an open oppertunity for each group to choose their own project and to present it at the end of the semester. My Team and I chose to create an Android application that is a mix between "rate my professor" and "reddit" where students rate a course and the ratings can be upvoted and downvoted (R.Edu = Rate my Education). I was in fully in charge of UI and partially connecting the controller (MVC) to the databse that my teamates had made.

\runsubsection{Apple Calculator}
\vspace{\topsep}
\begin{tightemize}
    \item An Android application developed to give users the ability to feel how the stock
    calculator on iOS feels.
    \item Scientific Calculator designed with modularity in mind where it enables other
    developers to add uniary and binary operations of their own.
    \item Implemented the Shunting-Yard algorithm to get the information from the user
    and used Postfix algorithm to grab that information and calculate it's the output.
    \item Made an abstract "token" object that can enable the creation of both "number" and
    "operation" objects so that both object types can be placed in one datastructure (due
    to the nature of the algorithm).
\end{tightemize}
\sectionsep

% A summer pitty project (that turned serious) where I thought of making a simple calculator and expected to finish in a day or two. The project turned out to take much longer then few days and it became much more then it let on. From learning new algorithms and techniques to a better tackle at android development due to it being a project with only myself.

\end{minipage}
\end{document}
